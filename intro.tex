\section{Introduction}
    Ce travail pratique porte sur l'étude des ressorts et des mouvements qu'ils réalisent 
    lorsqu'ils sont soumis à une charge, notamment le mouvement harmonique sinusoïdal.
    Nous analyserons également l'amortissement de ce mouvement en fonction du milieu dans 
    lequel évolue le système.  Les mesures ont été réalisées à l'aide d'un laser et du logiciel 
    CASSY Lab 2, mis à disposition pour enregistrer et analyser les données avec précision.
    
    \subsection{Experience Constante de Raideur}
    Premièrement, nous devrons déterminer la constante de raideur de trois ressorts différents. 
    Pour cela, nous effectuerons cinq mesures pour chaque ressort, en utilisant cinq surcharges 
    différentes, et nous mesurerons le déplacement du ressort par rapport à sa position de repos.

    \subsection{Experience Mouvement Harmonique Sinusoïdal (MHS)}
    Dans un second temps, nous utiliserons le ressort ayant la plus grande constante de raideur 
    ainsi que trois charges différentes pour déterminer la période d'oscillation, ainsi que l'amplitude 
    et la phase du mouvement. Nous confronterons ensuite les résultats obtenus expérimentalement à ceux 
    calculés théoriquement au préalable.

    \subsection{Experience Mouvement Harmonique Sinusoïdal Amortis (MHSA)}
    Dans un troisième temps, nous utiliserons le même ressort avec une seule surcharge pour effectuer trois 
    mesures consécutives de 120 secondes sans interrompre les oscillations. Ce protocole sera répété dans 
    trois milieux différents : d'abord dans l'air, puis dans l'eau, et enfin dans un mélange contenant 1/4 de 
    glycérine et 3/4 d'eau. Enfin, nous comparerons les résultats obtenus.
